\documentclass[10pt]{amsart} 


\usepackage{amsmath, amssymb, mathrsfs} 

\usepackage[mathscr]{euscript} 
 
\newlength{\mylength}
\setlength{\mylength}{0.25cm}

\usepackage{enumitem}
\setlist{listparindent=\parindent, itemsep=0cm, parsep=\mylength, topsep=0cm}

\usepackage[final]{todonotes}
\usepackage[final]{showkeys} 

\usepackage[breaklinks=true]{hyperref} 
\usepackage{comment} 

\usepackage{url}

\usepackage{tikz-cd}

\usepackage{amsthm}

\makeatletter
\renewenvironment{proof}[1][\proofname]{\par
	\pushQED{\qed}%
	\normalfont \topsep6\p@\@plus6\p@\relax
	\noindent\emph{#1.} 
	\ignorespaces
}{%
\popQED\endtrivlist\@endpefalse
}
\makeatother

\newtheoremstyle{mythm}% name of the style to be used
{\mylength}% measure of space to leave above the theorem. E.g.: 3pt
{0pt}% measure of space to leave below the theorem. E.g.: 3pt
{\itshape}% name of font to use in the body of the theorem
{0pt}% measure of space to indent
{\bfseries}% name of head font
{.\ }% punctuation between head and body
{ }% space after theorem head; " " = normal interword space
{\thmname{#1}\thmnumber{ #2}\thmnote{ (#3)}}

\newtheoremstyle{myrmk}% name of the style to be used
{\mylength}% measure of space to leave above the theorem. E.g.: 3pt
{0pt}% measure of space to leave below the theorem. E.g.: 3pt
{}% name of font to use in the body of the theorem
{0pt}% measure of space to indent
{\itshape}% name of head font
{.\ }% punctuation between head and body
{ }% space after theorem head; " " = normal interword space
{\thmname{#1}\thmnumber{ #2}\thmnote{ (#3)}}

\theoremstyle{mythm} 
%\newtheorem{thm}[subsubsection]{Theorem}
%\newtheorem*{claim}{Claim}
%\newtheorem*{thm}{Theorem} 
\newtheorem{thm}{Theorem}
\newtheorem{lem}[thm]{Lemma} 
\newtheorem{cor}[thm]{Corollary}
\newtheorem{claim}[thm]{Claim}
\newtheorem{prop}[thm]{Proposition}
%\newtheorem*{mthm}{Main Theorem}

%\newtheorem{prop}[subsubsection]{Proposition} 
%\newtheorem*{prop}{Proposition} 
%\newtheorem*{lem}{Lemma}
%\newtheorem*{klem}{Key Lemma}
%\newtheorem*{cor}{Corollary}

\theoremstyle{definition}
%\newtheorem{defn}[subsubsection]{Definition}
\newtheorem*{defn}{Definition} 
\newtheorem{prob}[thm]{Problem}
%\newtheorem{que}[subsubsection]{Question}

\theoremstyle{myrmk} 
%\newtheorem{rmk}[subsubsection]{Remark}
\newtheorem*{rmk}{Remark}
%\newtheorem{note}[subsubsection]{Note} 
\newtheorem*{ex}{Example}

\newcommand{\nc}{\newcommand} 
\nc{\on}{\operatorname}
\nc{\rnc}{\renewcommand} 

\rnc{\setminus}{\smallsetminus} 

\nc{\wt}{\widetilde}
\nc{\wh}{\widehat} 
\nc{\ol}{\overline} 

\nc{\Frob}{\on{Frob}}
\nc{\Gal}{\on{Gal}}

\nc{\BN}{\mathbb{N}}
\nc{\BZ}{\mathbb{Z}}
\nc{\BQ}{\mathbb{Q}}
\nc{\BR}{\mathbb{R}}
\nc{\BC}{\mathbb{C}}

\nc{\id}{\on{id}}
\nc{\Id}{\on{Id}}
\nc{\Tr}{\on{Tr}}

\nc{\la}{\langle}
\nc{\ra}{\rangle} 
\nc{\lV}{\lVert}
\nc{\rV}{\rVert}
\nc{\mb}{\mathbf}
\nc{\mf}{\mathfrak}
%\nc{\cur}{\mathscr}
\nc{\mc}{\mathscr}

\nc{\ira}{\hookrightarrow}
\nc{\hra}{\hookrightarrow}
\nc{\sra}{\twoheadrightarrow} 

\rnc{\Re}{\on{Re}}

\nc{\coker}{\on{coker}}
\nc{\End}{\on{End}}
\rnc{\Im}{\on{Im}}
%\rnc{\Re}{\on{Re}}

\nc{\Hom}{\on{Hom}}

\DeclareMathOperator*{\argmin}{arg\,min}
\DeclareMathOperator*{\argmax}{arg\,max}

\usepackage{marginnote}
\nc{\acts}{\curvearrowright}

\nc{\Mat}{\on{Mat}}

\newenvironment{cd}{\begin{equation*}\begin{tikzcd}}{\end{tikzcd}\end{equation*}\ignorespacesafterend}

\nc{\pfrac}[2]{\frac{\partial #1}{\partial #2}}
\nc{\e}[1]{\begin{align*} #1 \end{align*}}

\usepackage[margin=1in]{geometry}

\makeatletter
\def\blfootnote{\gdef\@thefnmark{}\@footnotetext}
\makeatother

%\renewcommand*{\arraystretch}{1.4}

\setlength{\parskip}{0.25cm}

\newenvironment{myproof}{\color{blue}\begin{proof}}{\end{proof}} 



\usepackage{fancyhdr}
\pagestyle{fancy} 
\fancyhead[L]{James Tao}
\fancyhead[C]{18.06 -- Week 8 Recitation}
\fancyhead[R]{Apr.\ 7, 2020}
\fancyfoot[C]{}

\newcounter{part-count}
\setcounter{part-count}{0}

\newenvironment{me}[1]{\begin{enumerate}[#1]\setcounter{enumi}{\value{part-count}}}{\setcounter{part-count}{\value{enumi}}\end{enumerate}}


\begin{document}
	\thispagestyle{fancy}
	
	\noindent Determine whether or not these objects exist. If so, write down an example. If not, explain why not. 
	
	\begin{me}{itemsep = 0.2cm}
		\item A $2 \times 2$ matrix $P$ satisfying $P^2 = P$, $\begin{pmatrix}
		1 \\ 1 
		\end{pmatrix} \in \on{col}(P)$, and $\begin{pmatrix}
		1 \\ 2
		\end{pmatrix} \in \on{null}(P)$. 
		\item An invertible matrix $V$ such that 
		\[
			V \begin{pmatrix}
			1 & 2 \\ 3 & 4 
			\end{pmatrix} V^{-1} = \begin{pmatrix}
			4 & 3 \\ 1 & 2
			\end{pmatrix}. 
		\]
		\item Two vectors $v, w \in \BR^2$ such that $v \cdot w = 1$ and $vv^\top + ww^\top = \on{Id}_{2 \times 2}$. 
		\item A square orthogonal matrix $Q$ such that $Q^3 = \on{Id}$, but neither $Q$ nor $Q^2$ equal the identity. 
		\item An upper triangular\footnote{This means that the entries of $U$ lying \emph{strictly} below the main diagonal are zero.} matrix $U$ such that $U^3 = \on{Id}$, but but neither $U$ nor $U^2$ equal the identity. 
		\item A square orthogonal matrix $Q$ such that $\det(Q) < 0$. 
		\item An orthogonal matrix such that $\det(QQ^\top) < 0$. 
		\item A symmetric matrix $A$ such that $\det(A) < 0$. 
		\item A matrix $A$ such that $\det(A^\top A) < 0$. 
		\item A real number $a$ such that the matrix 
		\[
			A = \begin{pmatrix}
			3 & a \\
			a & 1
			\end{pmatrix}
		\]
		transforms a shape with area 1 into a shape with area 4. 
		\item A $2 \times 2$ matrix which transforms the parallelogram with vertices $(1, 1), (2, -1), (-2, 1), (-1, -1)$ into a square of area 4. 
		\item A $2 \times 2$ matrix $A$ such that $\det(A) = 1$ and $A$ transforms the square with vertices $(1, 1), (1, -1), (-1, 1), (-1, -1)$ into a shape which lies inside the unit disk $D := \{v \in \BR^2 \text{ such that } \lVert v \rVert \le 1\}$. 
	\end{me}
	
	\newpage
	
	\section*{Solutions} 
	
	\noindent DNE stands for `does not exist.' 
	
	\begin{enumerate}
		\item $P = \begin{pmatrix}
		2 & -1 \\ 2 & -1
		\end{pmatrix}$. 
		\item DNE. We have 
		\[
			\det\left( V \begin{pmatrix}
			1 & 2 \\ 3 & 4 
			\end{pmatrix} V^{-1} \right) = \det(V) \cdot (-2) \cdot \det(V)^{-1} = -2, 
		\]
		which contradicts the RHS. 
		\item DNE. The second condition would imply 
		\[
			vv^\top v + ww^\top v = v. 
		\]
		The first condition implies $w^\top v = 1$, so 
		\[
			vv^\top v + ww^\top v = \lVert v \rVert^2 \, v + w. 
		\]
		Therefore $v$ and $w$ are linearly dependent. By a homework problem, this implies that 
		\[
			\on{rank}(vv^\top + ww^\top) < 2. 
		\]
		This contradicts $\on{rank}(\on{Id}_{2 \times 2}) = 2$. 
		\item Let $Q$ be a rotation matrix with angle $2\pi / 3$. 
		\item DNE. First, $U$ cannot be $1 \times 1$ because $U^3 = 1$ would imply that $U = 1$, contradiction. 
		
		Next, let $U$ be an $n \times n$ upper-triangular matrix. We prove, by induction on $n$, that $U^3 = \on{Id}_{n \times n}$ implies $U = \on{Id}_{n \times n}$. Fix $n \ge 2$ and assume the result holds for $n-1$. Decompose $U$ into blocks of these sizes: 
		\[
			\left( \begin{array}{c|c}
				(n-1)\times (n-1) & (n-1) \times 1 \\ \hline
				1 \times (n-1) & 1 \times 1
			\end{array} \right). 
		\]
		Since $U$ is upper-triangular, the bottom-left block is zero. This implies that the top-left block of $U^3$ is the cube of the top-left block of $U$. By the inductive hypothesis, the top-left block must equal $\on{Id}_{(n-1)\times (n-1)}$. Similarly, the bottom-right block of $U^3$ is the cube of the bottom-right block of $U$, so it equals 1. Thus, $U$ looks like 
		\[
			U =  \left(\begin{array}{c|c}
			\on{Id}_{(n-1) \times (n-1)} & b \\ \hline
			0_{1 \times (n-1)} & 1
			\end{array} \right)
		\]
		where $b$ is a $(n-1) \times 1$ matrix. Direct multiplication shows that 
		\[
			U^3 = \left( \begin{array}{c|c}
			\on{Id}_{(n-1) \times (n-1)} & 3b \\ \hline
			0_{1 \times (n-1)} & 1
			\end{array} \right), 
		\]
		so $U^3 = \on{Id}_{n \times n}$ implies that $b = 0$, hence $U = \on{Id}_{n \times n}$. This completes the induction. 
		\item Take $Q = (-1)$. 
		\item DNE. If $Q$ is square, then $\det(QQ^\top) = \det(Q)^2 \ge 0$. If $Q$ is $n \times m$ with $m < n$, then $\on{null}(Q^\top) > 0$, so $QQ^\top$ is not invertible, so its determinant equals zero. These are the only two possibilities for the size of $Q$. 
		\item Take $A = (-1)$. 
		\item DNE. If $\on{null}(A) > 0$, then $A^\top A$ is not invertible, so its determinant equals zero. If $\on{null}(A) = 0$, then we can write $A = QR$ where $Q$ is orthogonal and $R$ is invertible. Then 
		\[
			\det(A^\top A) = \det(R^\top R) = \det(R)^2 \ge 0. 
		\]
		\item Take $a = \sqrt7$, so $\det(A) = -4$. This matrix will transform a shape of are 1 into a shape of area 4. The negative sign on the determinant indicates that $A$ reverses orientation, i.e.\ it transforms a clockwise loop into a counterclockwise loop. 
		\item Take $A = \begin{pmatrix}
		\frac23 & \frac13 \\
		0 & 1
		\end{pmatrix}$. 
		\item DNE. The image of the given square under $A$ is a parallelogram which lies inside the unit disk $D$. The largest possible area of such a parallelogram is 2, which is attained by a square inscribed in the unit circle. Thus $\det(A) \le \frac12$. 
	\end{enumerate} 
	
\end{document} 