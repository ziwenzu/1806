\documentclass[10pt]{amsart} 


\usepackage{amsmath, amssymb, mathrsfs} 

\usepackage[mathscr]{euscript} 
 
\newlength{\mylength}
\setlength{\mylength}{0.25cm}

\usepackage{enumitem}
\setlist{listparindent=\parindent, itemsep=0cm, parsep=\mylength, topsep=0cm}

\usepackage[final]{todonotes}
\usepackage[final]{showkeys} 

\usepackage[breaklinks=true]{hyperref} 
\usepackage{comment} 

\usepackage{url}

\usepackage{tikz-cd}

\usepackage{amsthm}

\makeatletter
\renewenvironment{proof}[1][\proofname]{\par
	\pushQED{\qed}%
	\normalfont \topsep6\p@\@plus6\p@\relax
	\noindent\emph{#1.} 
	\ignorespaces
}{%
\popQED\endtrivlist\@endpefalse
}
\makeatother

\newtheoremstyle{mythm}% name of the style to be used
{\mylength}% measure of space to leave above the theorem. E.g.: 3pt
{0pt}% measure of space to leave below the theorem. E.g.: 3pt
{\itshape}% name of font to use in the body of the theorem
{0pt}% measure of space to indent
{\bfseries}% name of head font
{.\ }% punctuation between head and body
{ }% space after theorem head; " " = normal interword space
{\thmname{#1}\thmnumber{ #2}\thmnote{ (#3)}}

\newtheoremstyle{myrmk}% name of the style to be used
{\mylength}% measure of space to leave above the theorem. E.g.: 3pt
{0pt}% measure of space to leave below the theorem. E.g.: 3pt
{}% name of font to use in the body of the theorem
{0pt}% measure of space to indent
{\itshape}% name of head font
{.\ }% punctuation between head and body
{ }% space after theorem head; " " = normal interword space
{\thmname{#1}\thmnumber{ #2}\thmnote{ (#3)}}

\theoremstyle{mythm} 
%\newtheorem{thm}[subsubsection]{Theorem}
%\newtheorem*{claim}{Claim}
%\newtheorem*{thm}{Theorem} 
\newtheorem{thm}{Theorem}
\newtheorem{lem}[thm]{Lemma} 
\newtheorem{cor}[thm]{Corollary}
\newtheorem{claim}[thm]{Claim}
\newtheorem{prop}[thm]{Proposition}
%\newtheorem*{mthm}{Main Theorem}

%\newtheorem{prop}[subsubsection]{Proposition} 
%\newtheorem*{prop}{Proposition} 
%\newtheorem*{lem}{Lemma}
%\newtheorem*{klem}{Key Lemma}
%\newtheorem*{cor}{Corollary}

\theoremstyle{definition}
%\newtheorem{defn}[subsubsection]{Definition}
\newtheorem*{defn}{Definition} 
\newtheorem{prob}[thm]{Problem}
%\newtheorem{que}[subsubsection]{Question}

\theoremstyle{myrmk} 
%\newtheorem{rmk}[subsubsection]{Remark}
\newtheorem*{rmk}{Remark}
%\newtheorem{note}[subsubsection]{Note} 
\newtheorem*{ex}{Example}

\newcommand{\nc}{\newcommand} 
\nc{\on}{\operatorname}
\nc{\rnc}{\renewcommand} 

\rnc{\setminus}{\smallsetminus} 

\nc{\wt}{\widetilde}
\nc{\wh}{\widehat} 
\nc{\ol}{\overline} 

\nc{\Frob}{\on{Frob}}
\nc{\Gal}{\on{Gal}}

\nc{\BN}{\mathbb{N}}
\nc{\BZ}{\mathbb{Z}}
\nc{\BQ}{\mathbb{Q}}
\nc{\BR}{\mathbb{R}}
\nc{\BC}{\mathbb{C}}

\nc{\id}{\on{id}}
\nc{\Id}{\on{Id}}
\nc{\Tr}{\on{Tr}}

\nc{\la}{\langle}
\nc{\ra}{\rangle} 
\nc{\lV}{\lVert}
\nc{\rV}{\rVert}
\nc{\mb}{\mathbf}
\nc{\mf}{\mathfrak}
%\nc{\cur}{\mathscr}
\nc{\mc}{\mathscr}

\nc{\ira}{\hookrightarrow}
\nc{\hra}{\hookrightarrow}
\nc{\sra}{\twoheadrightarrow} 

\rnc{\Re}{\on{Re}}

\nc{\coker}{\on{coker}}
\nc{\End}{\on{End}}
\rnc{\Im}{\on{Im}}
%\rnc{\Re}{\on{Re}}

\nc{\Hom}{\on{Hom}}

\DeclareMathOperator*{\argmin}{arg\,min}
\DeclareMathOperator*{\argmax}{arg\,max}

\usepackage{marginnote}
\nc{\acts}{\curvearrowright}

\nc{\Mat}{\on{Mat}}

\newenvironment{cd}{\begin{equation*}\begin{tikzcd}}{\end{tikzcd}\end{equation*}\ignorespacesafterend}

\nc{\pfrac}[2]{\frac{\partial #1}{\partial #2}}
\nc{\e}[1]{\begin{align*} #1 \end{align*}}

\usepackage[margin=1in]{geometry}

\makeatletter
\def\blfootnote{\gdef\@thefnmark{}\@footnotetext}
\makeatother

%\renewcommand*{\arraystretch}{1.4}

\setlength{\parskip}{0.25cm}

\newenvironment{myproof}{\color{blue}\begin{proof}}{\end{proof}} 



\usepackage{fancyhdr}
\pagestyle{fancy} 
\fancyhead[L]{James Tao}
\fancyhead[C]{18.06 -- Linear independence, spans, and bases}
\fancyhead[R]{Mar.\ 6, 2020}
\fancyfoot[C]{}

\newcounter{part-count}
\setcounter{part-count}{0}

\newenvironment{me}{\begin{enumerate}\setcounter{enumi}{\value{part-count}}}{\setcounter{part-count}{\value{enumi}}\end{enumerate}}


\begin{document}
	\thispagestyle{fancy}

	Let $V$ be a vector space. For any nonempty subset $S \subseteq V$, we make the following definitions: 
	\begin{me}
		\item If every element of $V$ can be expressed as a linear combination of elements from $S$ in \underline{at most} one way, then $S$ is \emph{linearly independent}.\footnote{If $S$ is not linearly independent, we say that it is \emph{linearly dependent}. It would be more grammatically correct to say that the \emph{elements} of $S$ are linearly dependent or are linearly independent.} 
		\item If every element of $V$ can be expressed as a linear combination of elements from $S$ in \underline{at least} one way, then $S$ \emph{spans} $V$. We also say that $S$ is a \emph{spanning subset} of $V$. 
		\item \label{3} If both are true, then $S$ is a \emph{basis} of $V$. 
		\item $V$ is \emph{finite-dimensional} if it has a finite spanning subset. 
		
		In this case, any two bases of $V$ have the same (finite) size. This number is called the \emph{dimension} of $V$, denoted $\on{dim}(V)$. Any vector subspace of a finite-dimensional vector space is finite-dimensional. 
		\item The \emph{span} of $S$, denoted $\on{span}(S)$, is the subset of all linear combinations of elements of $S$. It is a vector subspace of $V$. 
	\end{me}
	Key facts about linear independence: 
	\begin{me}
		\item If $S \subseteq T$ and $T$ is linearly independent, then $S$ is linearly independent. 
		\item Any linearly independent subset of $V$ is contained in a basis of $V$. 
		\label{99} 
		
		Therefore, if $S$ is linearly independent, then $|S| \le \dim(V)$. 
		\item \label{9} The following statements about a nonempty subset $S \subseteq V$ are equivalent:  
		\begin{enumerate}[label=(\alph*)]
			\item $S$ is linearly independent. 
			\item Any nontrivial linear combination of elements in $S$ is nonzero. 
			\item $\dim(\on{span}(S)) = |S|$. 
			\item The span of any proper subset of $S$ is smaller than $\on{span}(S)$. 
			\item $S$ is a basis for the vector space $\on{span}(S)$. 
		\end{enumerate}
		\item If $S$ is linearly independent, and $v \in V$ is a vector, then $S \cup \{v\}$ is linearly independent if and only if $v \notin \on{span}(S)$. In this case, $\dim(\on{span}(S \cup \{v\})) = \dim(\on{span}(S)) +1$. 
	\end{me}
	Key facts about spanning subsets: 
	\begin{me}
		\item If $S \subseteq T$ and $S$ spans, then $T$ spans. 
		\item Any spanning subset of $V$ contains a basis of $V$. 
		 \label{13} 
		 
		 Therefore, if $S$ spans, then $|S| \ge \dim (V)$. 
	\end{me}
	Key facts about bases: 
	\begin{me}
		\item The following statements about a nonempty subset $S \subseteq V$ are equivalent:  
		\begin{enumerate}[label=(\alph*)]
			\item $S$ is a basis. 
			\item $S$ is a linearly independent subset of size $\dim (V)$. 
			
			(By~(\ref{99}), this is the largest possible size for a linearly independent subset.) 
			\item $S$ is a spanning subset of size $\dim(V)$. 
			
			(By~(\ref{13}), this is the smallest possible size for a spanning subset.) 
		\end{enumerate} 
		\item Let $\{v_1, \ldots, v_n\}$ be a basis. Then every element of $V$ can be expressed uniquely as a linear combination of $v_1, \ldots, v_n$, so there is a bijection $V \simeq \BR^n$ defined as follows: 
		\[
		(\lambda_1, \ldots, \lambda_n) \text{ in } \BR^n \qquad \text{ matches up with } \qquad \lambda_1v_1 + \cdots + \lambda_n v_n \text{ in } V,
		\]
		for any $\lambda_1, \ldots, \lambda_n \in \BR$. 
	\end{me}
	These concepts are related to the column space and null space of matrices. Let $A$ be an $n \times m$ matrix, and let $v_1, \ldots, v_m \in \BR^n$ be the columns of $A$. 
	\begin{me}
		\item \label{14} $\on{span}(\{v_1, \ldots, v_m\}) = \on{col}(A)$. 
		\begin{myproof}
			By definition, $\on{col}(A) \subseteq \BR^n$ is the set of all linear combinations of the columns of $A$, which are the $v_1, \ldots, v_m$. 
		\end{myproof}
		\item \label{15} 
		The following are equivalent: 
		\begin{enumerate}[label=(\alph*)]
			\item $v_1, \ldots, v_m$ are linearly independent. 
			\item $\on{rank}(A) = m$. 
			\item $\on{null}(A) = \{0\}$. 
		\end{enumerate} 
		\begin{myproof}
			We have 
			\e{
				\on{rank}(A) &= \dim(\on{col}(A)) \\
				&= \dim(\on{span}(\{v_1, \ldots, v_m\}))
			} 
			by (\ref{14}). Applying the statement (a) $\Leftrightarrow$ (c) in (\ref{9}), we conclude that $v_1, \ldots, v_m$ are linearly independent if and only if this number equals $m$. 
		\end{myproof} 
		\item \label{16} The following are equivalent: 
		\begin{enumerate}[label=(\alph*)]
			\item $v_1, \ldots, v_m$ are a spanning subset of $\BR^n$.  
			\item $\on{rank}(A) = n$. 
			\item $\on{col}(A) = \BR^n$. 
		\end{enumerate}
		\begin{myproof}
			$\on{rank}(A) = n$ if and only if $\on{col}(A) = \BR^n$. Thus the statement follows from~(\ref{14}). 
		\end{myproof}
		\item $v_1, \ldots, v_m$ are a basis if and only if $A$ is invertible. 
		\begin{myproof}
			By (\ref{3}), (\ref{14}), and (\ref{15}), we see that $v_1, \ldots, v_m$ are a basis if and only if $A$ is square and $\on{rank}(A) = n$. But this is true if and only if $A$ is invertible, as can be seen from the SVD. 
		\end{myproof}
	\end{me}
	\noindent Orthonormality implies linear independence: 
	\begin{me}
		\item If $v_1, \ldots, v_m \in \BR^n$ are an orthonormal collection of vectors, then they are linearly independent. 
		\begin{myproof}
			The $v_1, \ldots, v_m$ are the columns of an $n \times m$ orthogonal matrix $Q$. We've seen that $\on{rank}(Q) = m$, so the claim follows from (\ref{15}).  
		\end{myproof}
	\end{me}
	
	
\end{document} 