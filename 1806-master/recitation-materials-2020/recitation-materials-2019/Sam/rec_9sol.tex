\documentclass[11pt]{article}
\usepackage[hmargin=35pt,vmargin=35pt]{geometry}
\usepackage{graphicx}
\usepackage{amsfonts}
\usepackage{amsmath}
\usepackage{enumerate}
\pagenumbering{gobble} 
\newcommand{\diff}{\,\mathrm{d}}
\renewcommand*{\vec}[1]{\mathbf{#1}}

\title{18.06 - Recitation 9 - SOLUTIONS}
\author{Sam Turton}
\date{April 30, 2019}                                      
\begin{document}
\maketitle

\noindent \textbf{Problem 1.}\\
The matrix $B$ has eigenvalues $\lambda_1 = 2, \lambda_2 = 0$, and $\lambda_3 = 1$, with corresponding eigenvectors $x_1 = \begin{pmatrix} 1 \\ 2 \\ 1 \end{pmatrix}$, $x_2 = \begin{pmatrix} -2 \\ 1 \\ 0 \end{pmatrix}$, and $x_3 = \begin{pmatrix} 1 \\ 2 \\ -5 \end{pmatrix}$.
\begin{enumerate}
\item Find $B$ using the diagonalization formula $B = X\Lambda X^{-1}$. You can leave your answer as a product of the three matrices, as long as you write down each matrix explicitly (Hint: look at the eigenvectors. Finding $X^{-1}$ should require minimal computation).
\item Let $C=(I-B)(I+B)^{-1}$. What are the eigenvalues of C? (Hint: $B$ and $C$ have the same eigenvectors. Proving this will help you find the eigenvalues).
\end{enumerate}

\

\noindent \textbf{Solution}\\
\begin{enumerate}
\item $B = X\Lambda X^{-1}$, where 
$$\boxed{\Lambda = \begin{pmatrix} 2 & & \\ & 0 & \\ & & 1 \end{pmatrix}}$$
To find $X$ and $X^{-1}$ we can normalize the eigenvectors so that they form an orthonormal set:
$$q_1 = \frac{1}{\sqrt{6}}\begin{pmatrix} 1 \\ 2 \\ 1 \end{pmatrix}, \;\; q_2 = \frac{1}{\sqrt{5}}\begin{pmatrix} -2 \\ 1 \\ 0 \end{pmatrix}, \;\; q_3 = \frac{1}{\sqrt{30}} \begin{pmatrix} 1 \\ 2 \\ -5 \end{pmatrix}$$.
Then we can let 
$$X = \begin{pmatrix} \frac{1}{\sqrt{6}} & \frac{-2}{\sqrt{5}} & \frac{1}{\sqrt{30}} \\  \frac{2}{\sqrt{6}} & \frac{1}{\sqrt{5}} & \frac{2}{\sqrt{30}} \\  \frac{1}{\sqrt{6}} &0 & \frac{-5}{\sqrt{30}} \end{pmatrix},$$
so that $X$ is an orthogonal matrix for which $X^{-1} = X^T$. Then $B=X\Lambda X^T$. 
\item Let $Bx = \lambda x$. We can then show that:
\begin{align}
(I+B)x &= x+\lambda x = (1+\lambda) x\\
(I-B)x &= x - \lambda x = (1-\lambda) x
\end{align}
so that $x$ is an eigenvector of $I\pm B$. By inverting the first of these equations, we also have that $(I+B)^{-1} x = \frac{1}{1+\lambda} x$. Putting this all together we can then show that
$$Cx = (I-B)(I+B)^{-1} x = \frac{1-\lambda}{1+\lambda} x$$
So then the eigenvalues of $C$ are
$$ \frac{1-\lambda_1}{1+\lambda_1} = -\frac{1}{3}, \;\; \frac{1-\lambda_2}{1+\lambda_2} =1, \;\; \frac{1-\lambda_3}{1+\lambda_3} = 0$$
\end{enumerate}

\

\noindent \textbf{Problem 2.}\\
The matrix $A$ has diagonalization $A=X\Lambda X^{-1}$ with 
$$X = \begin{pmatrix} 1 & 1 & -1 & 0 \\ & 1 & 2 & 1 \\ & & 1 & 0 \\ & & & 1 \end{pmatrix}, \;\; \Lambda = \begin{pmatrix} 1 & & & \\ & 2 & & \\ & & -2 & \\ & & & -1 \end{pmatrix}.$$
Give a basis for the nullspace $N(M)$ of the matrix $M = A^4 - 2A^2 - 8I$.

\

\noindent \textbf{Solution}\\
The eigenvalues of $A$ are $\lambda = 1, 2, -2, -1$. The eigenvalues of $M$ are then $\lambda^4 - 2\lambda^2 - 8$, with the same corresponding eigenvectors. $M$ therefore has two zero eigenvalues (which come from the $\pm 2$ eigenvalues of $A$), and so the corresponding eigenvectors are a basis for $N(M)$, i.e.
$$\left\{\begin{pmatrix} 1 \\ 1 \\ 0 \\ 0 \end{pmatrix} ,\;\; \begin{pmatrix} -1 \\ 2 \\ 1 \\ 0 \end{pmatrix} \right\}$$

\

\noindent \textbf{Problem 3.}\\
Let $A,B,C$ and $D$ be $2\times 2$ matrices
\begin{enumerate}
\item Use the cofactor expansion to prove that the following block determinant expression holds:
$$\begin{vmatrix} A & 0 \\ C & D \end{vmatrix} = \vert A \vert \vert D \vert$$
\item Verify that if $A^{-1}$ exists, then
$$\begin{pmatrix} I & 0 \\ -CA^{-1} & I \end{pmatrix} \begin{pmatrix} A & B \\ C & D \end{pmatrix}  = \begin{pmatrix} A & B \\ 0 & D-CA^{-1} B\end{pmatrix} $$
\item Prove that
$$ \begin{vmatrix} A & B \\ C & D \end{vmatrix} = \vert AD - C B \vert $$
provided that $AC=CA$. 
\end{enumerate}

\

\noindent \textbf{Solution}\\
\begin{enumerate}
\item Let\footnote{You don't really need all the components to show this, but I figured it would be a little bit more transparent to write everything out.}
$$A=\begin{pmatrix} a & b \\ c & d \end{pmatrix} \;\; C = \begin{pmatrix} \alpha & \beta \\ \gamma & \delta \end{pmatrix}, \;\; D = \begin{pmatrix} u & v \\ w & x \end{pmatrix} $$
Then
\begin{align*}
\begin{vmatrix} A & 0 \\ C & D \end{vmatrix} &= \begin{vmatrix} a & b & 0 & 0 \\ c & d & 0 & 0 \\ \alpha & \beta & u & v \\ \gamma & \delta & w & x \end{vmatrix}\\
&= a \begin{vmatrix} d & 0 & 0 \\ \beta & u & v \\  \delta & w & x \end{vmatrix} - b\begin{vmatrix} c  & 0 & 0 \\ \alpha & u & v \\ \gamma & w & x \end{vmatrix}\\
&= ad  \begin{vmatrix}  u & v \\ w & x \end{vmatrix} - bc \begin{vmatrix}  u & v \\ w & x \end{vmatrix}\\
&= (ad-bc) \vert D \vert \\
&= \vert A \vert \vert D \vert
\end{align*}
Note that a similar process allows us to show that 
$$\begin{vmatrix} A & B \\ 0 & D \end{vmatrix} = \vert A \vert \vert D \vert$$
which we will need in the last part.
\item Multiplying out the left hand side gives the same as the right hand side (Be careful to do matrix multiplication in the correct order!)
\item Taking the determinant of both sides of the equation in part (2) shows that
\begin{align*}
\begin{vmatrix} I & 0 \\ -CA^{-1} & I \end{vmatrix} \begin{vmatrix} A & B \\ C & D \end{vmatrix}  &= \begin{vmatrix} A & B \\ 0 & D-CA^{-1} B\end{vmatrix} \\
\implies \begin{vmatrix} A & B \\ C & D \end{vmatrix} & = \vert A \vert \vert D-CA^{-1} B \vert \\
\implies \begin{vmatrix} A & B \\ C & D \end{vmatrix} & = \vert A \left(D-CA^{-1} B\right) \vert \\
&= \vert AD - ACA^{-1}B \vert\\
&= \vert AD - CB \vert
\end{align*}
where the last line holds provided that $AC=CA$. 
\end{enumerate}

\

\noindent \textbf{Problem 4.}\\
Recall that the matrix exponential of $A$ is defined via the infinite series
$$e^A = \sum_{n=0}^{\infty} \frac{A^n}{n!}.$$
\begin{enumerate}
\item Explain why $e^A$ is always an invertible matrix (hint: use eigenvalues). 
\item There is a result that says that whenever $AB=BA$, it holds that $e^{A+B}=e^{A}e^{B}$. Use this result to find the inverse of $e^{A}$. 
\item Suppose $A$ is a real, antisymmetric matrix so that $A^T=-A$. Show that $U= e^A$ is an orthogonal matrix.
\item If $x(t)$ satisfies 
$$\frac{dx}{dt} = Ax,$$
then explain why $\Vert x(t) \Vert = \Vert x(0) \Vert$ for all $t$.
\end{enumerate}

\

\noindent \textbf{Solution} \\
\begin{enumerate}
\item Suppose that $Ax = \lambda x$. Then we can show that 
$$e^A x =  \left(\sum_{n=0}^{\infty} \frac{A^n}{n!}\right) x = \sum_{n=0}^{\infty} \frac{A^n x}{n!} = \left(\sum_{n=0}^{\infty} \frac{\lambda^n }{n!} \right) x = e^{\lambda} x $$
So the eigenvalues of $e^A$ are given by $e^{\lambda}$ for each eigenvalue $\lambda$ of $A$. Hence $e^A$ can never have a zero eigenvalue (since $\vert e^{\lambda} \vert >0 \;\; \forall \lambda \in \mathbb{C}$). Therefore $e^A$ is always invertible for any matrix $A$.
\item Notice that 
$$e^{-A}e^{A} = e^{-A + A} = e^{0} = I$$
and so 
$$\left(e^A\right)^{-1} = e^{-A}$$
\item Notice that
$$U^T = \left(e^A\right)^T =  \left(\sum_{n=0}^{\infty} \frac{A^n}{n!}\right)^T = \left(\sum_{n=0}^{\infty} \frac{(A^T)^n}{n!}\right) = \left(\sum_{n=0}^{\infty} \frac{(-A)^n}{n!}\right) = e^{-A} = U^{-1}$$
and so $U^TU = I$. Therefore $U$ is an orthogonal matrix.
\item Recall that the solution of a matrix ODE $\frac{dx}{dt} = Ax$ is
$$x(t) = e^{At} x(0).$$
Therefore
$$\Vert x(t) \Vert ^2 = \left(e^{At} x(0)\right)^Te^{At} x(0) = x^T(0) \left(e^{At}\right)^T e^{At} x(0) = x^T(0) e^{-At} e^{At} x(0) = x^T(0) x(0) = \Vert x(0) \Vert^2 $$
\end{enumerate}

\

\noindent \textbf{Problem 5.}\\
A $3\times 3$ matrix $B$ is known to have eigenvalues $0,1,2$. This is enough information to determine 3 of the following. Which are true and what are their values:
\begin{enumerate}
\item The rank of $B$.
\item The determinant of $B^TB$.
\item The eigenvalues of $B^TB$.
\item The eigenvalues of $(B^2+I)^{-1}$. 
\end{enumerate}

\

\noindent \textbf{Solution} \\
\begin{enumerate}
\item Since $B$ has 3 distinct eigenvalues, exactly one of which is 0, we know that the dimension of the nullspace is 1. Since $B$ is $3\times 3$, we can deduce that $r = 3-1 = 2$ .
\item The determinant of $B^TB$ is 
$$\vert B^TB \vert = \vert B^T \vert \vert B\vert = \vert B \vert^2$$
But we know that $\vert B \vert = 0$, since the determinant of a matrix is the product of its eigenvalues. Therefore $\vert B^TB \vert = 0$.
\item The eigenvalues of $B^TB$ cannot be determined from this information.
\item Suppose $Bx = \lambda x$ so that $\lambda$ is an eigenvalue of $B$ with eigenvector $x$. Then
$$(B^2 + I)x = B^2x + x = \lambda^2 x + x = (\lambda^2 +1)x$$
and so $x$ is also an eigenvector of $B^2+I$, but with eigenvalue $\lambda^2 + 1$. By inverting this equation, we then also have that
$$(B^2+I)^{-1} x = \frac{1}{\lambda^2 +1}.$$
Hence if $B$ has eigenvalues $0,1,2$, then $(B^2+I)^{-1}$ must have eigenvalues $1,1/2, 1/5$.
\end{enumerate}


\end{document}  